\documentclass[11pt]{exam}
\usepackage[activeacute,spanish]{babel} % Permite el idioma espa\~nol.
\usepackage[utf8]{inputenc} 
\usepackage{amsmath}
\usepackage[colorlinks]{hyperref}
\usepackage{graphicx}
\usepackage[amssymb]{SIunits}
\pagestyle{headandfoot}
\spanishdecimal{.}

\begin{document}
\sffamily

\firstpageheadrule
%\firstpagefootrule
\firstpagefooter{}{Pagina \thepage\ de \numpages}{}
\runningheadrule
%\runningfootrule
%\lhead{\bf\normalsize  Relatividad General\\ Tarea 01, 2022-1}
%\rhead{\bf\normalsize Ciencias F\'isicas \\ Astronomía}
\chead{\bf\normalsize Depto. de F\'isica \\ Universidad de Concepci\'on}
\rfoot{\thepage\ / \numpages}
\cfoot{ }
\lfoot{\tiny }
\begin{flushleft}
\vspace{0.2in}
%\hbox to \textwidth{Nombre: \enspace \hrulefill}
%Nombre : \\
\vspace{0.25cm}
\end{flushleft}
%%%%%%%%%%%%%%%%%%%%%%%%%%%%%%%%%%%%%%%%%%
\begin{center}
\textbf{Actividad 2: Módulo lunar}
\end{center}

La siguente actividad está diseñada para mostrar la segunda ley de Newton en acción, con el ejemplo de un cuerpo en movimiento en 1D bajo la acción de la gravedad, y frenado por la resistencia del aire. \\

Al final de esta actividad, el/la estudiante será capaz de: graficar datos de aceleración versus tiempo, obtener valores de fuerza a partir de valores de aceleración, calcular valores de fuerza de roce versus tiempo.\\

Conceptos clave: Segunda ley de Newton; fuerza de gravedad; Fuerza de roce.\\

\begin{questions}
\item Queremos modelar el movimiento de un módulo lunar que reingresa a la atmósfera. Este movimiento es afectado por la fuerza de gravedad sobre el módulo, $\vec{F}_g$, y la fuerza de resistencia provocada por el aire durante la caída, $\vec{F}_r$ (ver Figura \ref{dcl}). La masa del módulo es de 5000 kg. En el archivo de datos modulo.txt se listan los valores de aceleración $a(t_i)$ para distintos tiempos $t_i$, para dicho módulo. Con estos datos, realice las siguientes tareas:

\begin{parts}
\item  Grafique los datos de aceleración v/s tiempo.
%\item Grafique la velocidad.
\item Escriba la segunda ley de Newton para esta situación y despeje la resistencia del aire.
\item  Grafique los valores de la resistencia del aire en función del tiempo.
\item El módulo puede dañarse si se expone a fuerzas de resistencia del aire mayores a $10^6$ N por más de 5 segundos. Basado en el resultado del punto anterior, ¿qué pasará con el módulo? Justifique.
\end{parts}
\begin{center}
\begin{figure}[h]
    \centering
    \includegraphics[height=0.26\textwidth]{dcl_modulo.png}    
    \caption{Diagrama de cuerpo libre del módulo.}
    \label{dcl}
\end{figure}
  
\end{center}


\end{questions}

\end{document} 
