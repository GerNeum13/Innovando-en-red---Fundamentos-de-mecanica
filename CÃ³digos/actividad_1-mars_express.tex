\documentclass[11pt]{exam}
\usepackage[activeacute,spanish]{babel} % Permite el idioma espa\~nol.
\usepackage[utf8]{inputenc} 
\usepackage{amsmath}
\usepackage[colorlinks]{hyperref}
\usepackage{graphicx}
\usepackage[amssymb]{SIunits}
\pagestyle{headandfoot}
\spanishdecimal{.}

\begin{document}
\sffamily

\firstpageheadrule
%\firstpagefootrule
\firstpagefooter{}{Pagina \thepage\ de \numpages}{}
\runningheadrule
%\runningfootrule
%\lhead{\bf\normalsize  Relatividad General\\ Tarea 01, 2022-1}
%\rhead{\bf\normalsize Ciencias F\'isicas \\ Astronomía}
\chead{\bf\normalsize Depto. de F\'isica \\ Universidad de Concepci\'on}
\rfoot{\thepage\ / \numpages}
\cfoot{ }
\lfoot{\tiny }
\begin{flushleft}
\vspace{0.2in}
%\hbox to \textwidth{Nombre: \enspace \hrulefill}
%Nombre : \\
\vspace{0.25cm}
\end{flushleft}
%%%%%%%%%%%%%%%%%%%%%%%%%%%%%%%%%%%%%%%%%%
\begin{center}
\textbf{Actividad 1: Mars Express}
\end{center}


La siguente actividad está diseñada para  demostrar cómo analizar las posiciones, desplazamientos, velocidades y aceleraciones medias a partir de datos reales de posición, utilizando Python. \\

Al final de esta actividad, el/la estudiante será capaz de: calcular y graficar vectores de desplazamiento, calcular norma de vectores, calcular angulo entre vectores.\\

Conceptos clave: Desplazamiento; velocidad; aceleración; vectores. \\

\begin{questions}
\item En la página de Canvas de la práctica se encuentra el archivo de datos a analizar, marsexpress.txt, un notebook de Jupyter con los códigos mostrados en la práctica, ME1.ipynb, y la versión pdf del resumen de la práctica. Con estos archivos, realice las siguientes tareas:

\begin{parts}
\item  Calcule los (6) vectores de desplazamiento del Mars Express. 
%\item Grafique la velocidad.
\item Grafiquer los (6) vectores de desplazamiento del Mars Express.
\item  Calcule la norma o módulo de cada vector desplazamiento.
\item Calcule el ángulo que cada vector desplazamiento forma con el eje . 
\end{parts}
\end{questions}

\end{document} 
